%
% $Id: console.tex,v 1.5 1998/02/09 22:59:30 swa Exp $
%
\NeedsTeXFormat{LaTeX2e}

\documentclass{article}

\addtolength{\parskip}{1ex plus0.5ex minus0.2ex}

\addtolength{\textwidth}{2cm}

\begin{document}

\section{The console program}

Version of this document is
\begin{verbatim}
$Id: console.tex,v 1.5 1998/02/09 22:59:30 swa Exp $
\end{verbatim}

\subsection{What is console?}

Suppose you want to run a couple of programs on several machines. You would
then log into one machine, start one program, log into another machine, start
another program and so forth. You would start various xterms and clobber your
screen with a lot of windows. Well, this is very time consuming and there is a
solution. With console you can start multiple programs on remote machines and
display the output in one window. Console can even take care of dependencies
between these programs. For example, if program A must be started before
program B, console will automatically launch program A, once you started
program B.

\subsection{Requirements}

Console is a C program that has a Tcl/Tk front end. Thus you need to install
Tcl/Tk and some additional software to run the console program. Fortunately,
installing has become quite easy these days, so it shouldn't be too difficult.
Console was developed and tested with the following RedHat packages:

\begin{enumerate}
\item \texttt{tcl-7.6pl2-3.i386.rpm}
\item \texttt{tk-4.2pl2-3.i386.rpm}
\item \texttt{tix-4.1.0rel-3.i386.rpm}
\item \texttt{blt-2.3-1.i386.rpm}
\end{enumerate}

If you don't have a RedHat system, get the stuff from these locations:

\begin{enumerate}
\item \texttt{http://sunscript.sun.com/}
\item \texttt{http://www.xpi.com/tix/}
\item \texttt{http://www.tcltk.com/blt/index.html}
\end{enumerate}

Make sure, that you download the appropriate versions corresponding to the
RedHat packages I used.

Console starts programs on remote hosts. One way to do this (there are other
ways, e.g. rsh) is with \texttt{ssh}, which I use in the example.  Both rsh
and ssh typically don't require the user to enter his username and password;
they use different means for authentication. For that to work, check out ssh's
or rsh's man page; in particular you need for rsh

\begin{itemize}
\item on every remote host a file \texttt{\~{}/.rhosts}, which lists the
  machines (one per line) that have access to the remote host thru rsh.
\item \texttt{\~{}/.rhosts} must be world-readable.
\end{itemize}

and for ssh (which I highly recommend)

\begin{itemize}
  
\item Get it and install it as root with 
  \begin{center}
    \small
    \texttt{rpm -Uvh
      ftp://ftp.replay.com/pub/replay/pub/redhat/i386/ssh-1.2.20-4i.i386.rpm}
  \end{center}
  and 
  \begin{center}
    \small
    \texttt{rpm -Uvh
      ftp://ftp.replay.com/pub/replay/pub/redhat/i386/ssh-clients-1.2.20-4i.i386.rpm}.
  \end{center}
  Or retrieve the latest version that can be found at
  \begin{center}
    \texttt{http://www.replay.com/redhat/}
  \end{center}
\item Read the documentation at
  \begin{center}
    \texttt{http://www.cs.hut.fi/ssh/}
  \end{center}
  in particular
  \begin{center}
    \texttt{http://www.tac.nyc.ny.us/\~{}kim/ssh/}
  \end{center}
\end{itemize}

\subsection{How the console program works}

Initially, the program parses the file \texttt{beeSoft.ini} which defines
various parameters and dependencies among beeSoft's modules. If you have
written your own program that interacts with these modules, you would define
it's parameters and dependencies in the file \texttt{console.ini}.

Once you started console, it will display all programs that you defined with
the variable \texttt{myPrograms} in \texttt{console.ini}. Console will also
include all the other programs that might be required to run the programs
defined by \texttt{myPrograms}.

Clicking on one button will start the program and all it's dependent programs.
Once the program is running, the button changes to a `Stop' button. If you
click the `Stop' button it will kill only this program and \emph{none} of it's
dependent programs. However, killing for example the tcxServer will likely
bring down all other programs.

There are some menu points, that you might want to try out.
\texttt{Options|Display parsed data...} shows the data, that console got from
\texttt{beeSoft.ini} and \texttt{console.ini}. \texttt{Help|About...} and
\texttt{File|Quit} do what they promise.

\subsection{How to use console}

First of all, you have to modify \texttt{beeSoft.ini} to reflect your setup of
machines.

\begin{enumerate}
\item Check on a shell, if you can use ssh/rsh to execute programs on a remote
  machine.
\item For each program in \texttt{beeSoft.ini}, change the entry
  \texttt{programOnHost} to the appropriate machine. If console finds a
  \texttt{***} in \texttt{programOnHost}, it will try to pick the \texttt{host}
  entry from the hardware device.
  
  Example: you want to configure the camera server: its \texttt{programOnHost}
  contains \texttt{***} --- so far so good.  Now look for
  \texttt{framegrabber} in the \texttt{[robot]} section.  It should say
  \texttt{matrox}. To find the host on which console would run the camera
  server, look at the \texttt{host} entry in the section
  \texttt{[matrox.framegrabber]}. I'm sure you get the point.

\item Do not change anything else. It took me a fairly long and painful time
  to figure out the right combination of commands, pipes and ampersands.
\end{enumerate}

Now for your own program (let's call it gestureServer), write a
\texttt{console.ini} file. If console cannot find a \texttt{console.ini}, it
will exit. To be precise: If console cannot find the entry
\texttt{pathToMyPrograms} which it is supposed to find in
\texttt{console.ini}, it will exit. Here is the syntax of the file:

\begin{description}
  
\item \texttt{[bParamFile]}
  
  This must be the first line of the file, so that console can identify the
  file as an initialization file.

\item \texttt{myPrograms}
  
  Is a comma separated list of programs that the console program will use. For
  each item in this list, it will also use it's dependent programs. Example:
  \begin{center}
    \texttt{ myPrograms \{ gestureServer, recognizeServer \} }
  \end{center}
  will use \texttt{gestureServer} and \texttt{recognizeServer}.
  
\item \texttt{pathToMyPrograms} 
  
  Is a list of paths where the console program looks for beeSoft's and your
  programs. Example:
  \begin{center}
    \texttt{ pathToMyPrograms /home/swa/bee/bin:/home/bee/bin }
  \end{center}
  will look for programs first in \texttt{/home/swa/bee/bin} and then in
  \texttt{/home/bee/bin}. Listing two entries here comes in handy if you
  prefer your local binary over the global one.

\item \texttt{scrollauto}
  
  If set to \texttt{1} will automatically scroll to the end of each log
  window.

\item \texttt{[gestureServer]}
  
  Is the start of gestureServer's definition.
  
\item \texttt{name}
  
  Is a symbolic name of the program. In this case we use \texttt{gesture},
  simply because it's shorter and more convenient than \texttt{gestureServer}.

\item \texttt{programOnHost}

  Defines the host, on which we run the program \texttt{gestureServer}. Don't
  use \texttt{***} here, because console will likely not find a definition in 
  \texttt{beeSoft.ini}.

\item \texttt{howStart1, howStart2, howStart3}
  
  Defines how to start the program on the remote machine. In the case of the
  gesture server, we use

\begin{verbatim}
howStart1       /usr/bin/ssh
howStart2       -n
howStart3       gestureServer
\end{verbatim}
  
  It's generally not a good idea to change these parameters. There are a lot
  of ways in which \texttt{stdout} or \texttt{stderr} might not get
  transmitted back to our machine, etc. Just change \texttt{gestureServer} to
  the name of your program. The program will be picked out of
  \texttt{pathToMyPrograms}
  and started with 
  \begin{center}
    \small
    \texttt{/bin/sh -c 'PATH=/home/swa/bee/bin:/home/bee/bin gestureServer 2>\&1'}
  \end{center}


\item \texttt{howKill1, howKill2, howKill3}

  Defines how to kill the program on the remote machine. In the case of the
  gesture server, we use

\begin{verbatim}
howKill1        /usr/bin/ssh
howKill2        -n
howKill3        /bin/sh -c 'killall gestureServer 2>&1'
\end{verbatim}

  Again, don't change these parameters, just change \texttt{gestureServer} to
  the name of your program.

\item \texttt{dep}

  Is a comma separated list of programs that must be started before this
  program can proceed. Example:
  \begin{center}
    \texttt{ dep \{ tcxServer, cameraServer \} }
  \end{center}
  will use the definition of \texttt{tcxServer} and \texttt{cameraServer}.

\end{description}

\end{document}