%
% $Id: laserServer.tex,v 1.5 1998/02/03 18:06:29 swa Exp $
%
\NeedsTeXFormat{LaTeX2e}

\documentclass{article}

\addtolength{\parskip}{1ex plus0.5ex minus0.2ex}

\addtolength{\textwidth}{2cm}

\begin{document}

\section{The laser server}

Version of this document is
\begin{verbatim}
$Id: laserServer.tex,v 1.5 1998/02/03 18:06:29 swa Exp $
\end{verbatim}

\subsection{How the laser server works} 
 
A program dealing with the laser server typically has four steps (take a look at
\begin{center} 
  \texttt{\~{}bee/src/laserServer/laserExample.c}
\end{center}
for an example):

\begin{enumerate}
  
\item \texttt{laserRegister( );} 
  
  Registers all new commands that we need for dealing with the laser server.
  
\item \texttt{laserConnect( 1 );} 
  
  Waits until we have connected to the laser server.
  
\item \texttt{registerLaserSweepCallback( myLaserSweepCallback );} 
  
  Registers the function \texttt{myLaserSweepCallback}, which is called once
  the laser has finished a full sweep.

\item \texttt{laserSubscribeSweep( 0, 1 );}
  
  We subscribe to every sweep of the first laser (the first laser is
  \texttt{0}). Every time the first laser has finished a full sweep, the laser
  server calls the function \texttt{myLaserSweepCallback}.

\end{enumerate}
 
If you want to use the laser server, follow the above four steps, include
\texttt{laserClient.h} in your source and link with \texttt{-llaserClient}.

\subsection{Commands for the laser server}

\begin{description}
  
\item \texttt{void laserRegister( );}

  Registers the laser server.
 
\item \texttt{int laserConnect( int wait\_till\_established );} 
  
  Connects to the laser server. If \texttt{wait\_till\_established} is
  \texttt{1}, we wait until connection with the laser server has been
  established. Otherwise we proceed.
 
\item \texttt{void registerLaserSweepCallback( laserSweepCallbackType fcn );}
  
  Registers a function \texttt{fcn}. The laser server calls this function
  every time it has received a new sweep from the laser.

\item \texttt{void laserRequestSweep( int numLaser );}
  
  Requests a laser sweep from laser \texttt{numlaser}. The laser server calls
  the function \texttt{fcn} that was previously registered with
  \texttt{registerLaserSweepCallback(\dots)}.

\item \texttt{void laserSubscribeSweep( int numLaser, int number );}
  
  If \texttt{number} is not zero, subscribe to the every \texttt{number}-th
  sweep of the laser \texttt{numlaser}. If \texttt{number} is zero,
  unsubscribe. Every time the laser finishes a sweep, the laser server calls
  the function \texttt{fcn} that was previously registered with
  \texttt{registerLaserSweepCallback(\dots)}.

\end{description}

\subsection{Parameters for the laser server}

\begin{description}

\item \texttt{-display=0}
  
  Do not display the laser sweep. This is the default.

\item \texttt{-display=1}
  
  Open a window and display the current laser sweep.

\end{description}

\subsection{\texttt{beeSoft.ini} and the laser server}

The file \texttt{beeSoft.ini} contains some configuration for the laser
server.  In the section \texttt{[robot]} you find an entry \texttt{laser}
which should either say \texttt{SICK} or \texttt{none}. Currently, the
SICK laser is the only supported laser.

If you have a SICK laser, specify `\texttt{laser SICK}' and look at the
section \texttt{[SICK.laser]}. The \texttt{type} entry should say
\texttt{SICK PLS101}.

The laser server does not currently use the \texttt{host} entry. \texttt{dev}
specifies the device, which the laser server uses to communicate with the
laser (typically \texttt{/dev/cur6}). The only supported baudrate \texttt{bps}
at this time is \texttt{9600}.

\subsection{The example program \texttt{laserExample}}
 
Start \texttt{tcxServer} and \texttt{laserServer} (the laser server of
course on the machine that has control over the laser). Now start the program
\texttt{laserExample}.

The program will connect to the laser server, register a callback function and
subscribe to every laser reading. Upon receiving the sweep in
\texttt{myLaserSweepCallback}, it will print out the reading. You can kill and
restart both the server (laserServer) and the client (laserExample), they will
reconnect to each other.

\end{document}