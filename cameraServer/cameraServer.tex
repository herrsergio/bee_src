%
% $Id: cameraServer.tex,v 1.9 1998/01/13 00:35:13 swa Exp $
%
\NeedsTeXFormat{LaTeX2e}

\documentclass{article}

\addtolength{\parskip}{1ex plus0.5ex minus0.2ex}

\addtolength{\textwidth}{2cm}

\begin{document}

\section{The camera server}

Version of this document is
\begin{verbatim}
$Id: cameraServer.tex,v 1.9 1998/01/13 00:35:13 swa Exp $
\end{verbatim}
 
\subsection{How the camera server works} 

We assume that you have at least one Matrox Meteor frame grabber and a working
driver for this card. If the driver is not installed, read the latest
BeeSoft-README file for installing the driver.
 
Once you start the camera server, it will open one or more devices as
specified in \texttt{beeSoft.ini}. You will find out more about
\texttt{beeSoft.ini} in a later section.
 
The size of the frame (\texttt{ROWS} and \texttt{COLS}) is set in
\texttt{\~{}bee/src/cameraServer/imagesize.h}, by default 240 rows by 320
columns. We set the frame size at compile time for performance and simplicity
reasons. Each frame consists of \texttt{ROWS*COLS} pixel, each pixel of 4
bytes (B, G, R -- the 4th byte per pixel is only used to obey word boundaries
and can be omitted).

The camera server will store the frame in a shared memory segment. To access
the frame from your program, you have two options.

\begin{enumerate}
  
\item Thru TCX, which means that the images are sent (maybe sub-sampled) over
  the network to whatever machine you like. This method has the advantage,
  that you know exactly \emph{when} the frame grabber has finished grabbing an
  image.  However, the camera server sends the images over the network, which
  is very slow.
  
\item Thru a shared memory segment. This requires that you program runs on the
  same computer (we have to share the same memory, that's why it's called
  shared memory). The shared memory segment is identified by an ID, which your
  program needs to access this segment. The major advantage of this method is
  it's speed. The catch however is, that you don't know when the frame grabber
  has finished grabbing a particular frame. The camera server keeps updating
  the shared memory segment at roughly 30 frames per second. Is is highly
  unlikely, that your program will also process frames 30 times per second.

\end{enumerate}
 
In the following section we will give a brief overview over the
shared-memory-segment method. The other method is left as an exercise to the
reader (it's too simple and too slow anyway).
 
A program dealing with the camera server typically has six steps (take a look at
\begin{center} 
  \texttt{\~{}bee/src/cameraServer/mainatt.c}
\end{center}
for an example):

\begin{enumerate}
  
\item \texttt{cameraRegister( );} 
  
  Registers all new commands that we need for dealing with the camera server.
  
\item \texttt{cameraConnect( 1 );} 
  
  Waits until we have connected to the camera server.
  
\item \texttt{registerCameraShmIdCallback( myCameraShmIdCallback );} 
  
  Registers the function \texttt{myCameraShmIdCallback}, which is called once
  we receive the ID of the shared memory segment. Remember, we cannot access
  the shared memory segment and thus the image without the ID.
  
\item \texttt{cameraRequestShmId( int numGrabber );} 
  
  Requests the ID for frame grabber \texttt{numGrabber}. The camera server
  (once it has connected to that frame grabber and established the shared
  memory segment) will call \texttt{myCameraShmIdCallback} with the ID and the
  number of the frame grabber as parameters.
  
\item We wait until \texttt{myCameraShmIdCallback} is called and if it is
  called, attach to the shared memory segment.

\item Process the shared memory segment.

\end{enumerate}
 
If you want to use the camera server, follow the above six steps, include
\texttt{cameraClient.h} in your source and link with \texttt{-lcameraClient}.

\subsection{Commands for the camera server}

\begin{description}
  
\item \texttt{void cameraRegister( );}

  Registers the camera server.
 
\item \texttt{int cameraConnect( int wait\_till\_established );} 
  
  Connects to the camera server. If \texttt{wait\_till\_established} is
  \texttt{1}, we wait until connection with the camera server has been
  established. Otherwise we proceed.
 
\item \texttt{void registerCameraImageCallback( cameraImageCallbackType fcn
    );} 
  
  Registers a function \texttt{fcn}. The camera server calls this function
  every time it receives a new frame from one of the frame grabbers. Keep in
  mind that these frames are sent to the requesting program via TCX.
 
\item \texttt{void registerCameraShmIdCallback( cameraShmIdCallbackType fcn
    );} 
  
  Registers a function \texttt{fcn}. The camera server calls this function
  once it has established it's shared memory segment with one or more of
  the frame grabbers.
 
\item \texttt{void registerCameraFileCallback( cameraFileCallbackType fcn );}
  
  Registers a function \texttt{fcn}. The camera server calls this function
  every time it has finished a file operation. Not yet implemented.
 
\item \texttt{void cameraRequestShmId( int numGrabber );} 
  
  Requests the ID of the shared memory segment for frame grabber
  \texttt{numGrabber}.
 
\item \texttt{void cameraRequestImage( int numGrabber, int xsize, int ysize
    );}
  
  Requests a frame of size \texttt{xsize} times \texttt{ysize} pixel from
  frame grabber \texttt{numGrabber}. The size may not exceed \texttt{ROWS} and
  \texttt{COLS} as defined in \texttt{\~{}bee/src/cameraServer/imagesize.h.}
  This frame will be sent to the requesting module via TCX, i. e. the camera
  server calls the function, that you have registered with
  \texttt{registerCameraImageCallback}.
 
\item \texttt{void cameraSubscribeImage( int numGrabber, int number, int
    xsize, int ysize );}
  
  If \texttt{number} is not zero, subscribe to every \texttt{number}-th frame
  from the the camera server. If \texttt{number} is zero, unsubscribe. This
  frame from frame grabber \texttt{numGrabber} will be sent to the requesting
  module via TCX, i. e. the camera server calls the function, that you have
  registered with \texttt{registerCameraImageCallback}.
 
\item \texttt{void cameraStartSaving( int numGrabber, char *filename );} 
  
  Start saving continuously to file \texttt{filename} from frame grabber
  \texttt{numGrabber}. The file will be written on the machine that runs the
  camera server.
 
\item \texttt{void cameraStopSaving( int numGrabber );} 
  
  Stop the saving process of frame grabber \texttt{numGrabber}.
 
\item \texttt{void cameraSaveFile( int numGrabber, char *filename, int num );} 
  
  Start saving \texttt{num} frames from frame grabber \texttt{numGrabber} to
  file \texttt{filename}. The file will be written on the machine that runs
  the camera server.
 
\item \texttt{void cameraLoadFile( int numGrabber, char *filename );} 
  
  Loads file \texttt{filename} on the machine that runs the camera server.
  The camera server treats the images from the file as if they were received
  by frame grabber \texttt{numGrabber}.

\item \texttt{void cameraRestartGrabber( int numGrabber );} 
  
  Restarts frame grabber \texttt{numGrabber} after a stop.

\item \texttt{void cameraStopGrabber( int numGrabber );} 
  
  Stops frame grabber \texttt{numGrabber} from continously grabbing images.

\end{description}

\subsection{Parameters for the camera server}

The parameters \texttt{display} and \texttt{color} work for one and two frame
grabbers.

\begin{description}

\item \texttt{-display=0}
  
  Do not display the camera image. This is the default.

\item \texttt{-display=1}
  
  Open a window and display the current image. If you use more than one frame
  grabber, setting \texttt{-display=1} can have strange effects. Your X server
  is probably not able to display the images as fast as they come in. It's not
  a bad idea, to set \texttt{-display=0} when using more than one frame grabber.

\item \texttt{-tcx=1}
  
  Use TCX. This is the default.

\item \texttt{-tcx=0}
  
  Do not use TCX. Used only for testing the frame grabber.

\item \texttt{-color=0}
  
  Display the image in gray scale on an 8 bit display. This is the default.

\item \texttt{-color=1}
  
  Display the image in color on an 8 bit display. Although we grab images at
  24 bit depth, we display them in 8 bit depth. Thus the colors in the image
  are only approximations of the `real' colors.

\end{description}

\subsection{\texttt{beeSoft.ini} and the camera server}

The file \texttt{beeSoft.ini} contains some configuration for the camera
server.  In the section \texttt{[robot]} you find an entry
\texttt{framegrabber} which should either say \texttt{matrox} or
\texttt{none}. Currently, the Matrox Meteor is the only supported frame
grabber.  

In case you want to use the camera server on a machine which does not have a
frame grabber, say \texttt{none}. However, you can still use the camera server
to load images from disk.

If you have a Matrox Meteor, specify `\texttt{framegrabber matrox}' and look
at the section \texttt{[matrox.framegrabber]}. The \texttt{type} entry should
say \texttt{matrox-meteor}.

The entry \texttt{usegrabber} specifies which frame grabber the camera server
will use. If you say \texttt{0}, no frame grabber will be used --- however,
you can still use the camera server to load images from disk. \texttt{1} uses
the first frame grabber, \texttt{2} uses the second frame grabber and
\texttt{3} uses both frame grabbers.

Last but not least, you can specify, which devices the camera server should
use to access the frame grabbers. Typically \texttt{dev1} is
\texttt{/dev/mmetfgrab0} and \texttt{dev2} is \texttt{/dev/mmetfgrab1}.

\subsection{The example program \texttt{cameraAttachExample}}
 
Start \texttt{tcxServer} and \texttt{cameraServer} (the camera server of
course on the machine that has the frame grabber). Now start the program
\texttt{cameraAttachExample} on the same machine (we share the same memory) as
the camera server.

If you run your X server in 8 bit mode, the program should simply display a
window with the current camera image. In case you told the camera server to
use more than one frame grabber (in \texttt{beeSoft.ini}), it will display
both camera images.

If you press the left mouse button within that window, the program will write
the current camera image to a file \texttt{image-00.ppm} or
\texttt{image-01.ppm}, depending on the source. You can use for example
\texttt{xv} to display \texttt{image-\{00|01\}.ppm} (or use \texttt{xv} to
convert \texttt{image\{00|01\}.ppm} into whatever format you like).
 
\subsection{The program \texttt{cameraControl}}
 
Start \texttt{tcxServer} and \texttt{cameraServer} (the camera server of
course on the machine that has the frame grabber or grabbers). Now start the
program \texttt{cameraControl}.

\texttt{cameraControl} is a C program with a Tcl/Tk user interface that
remotely controls the camera server. You can use \texttt{cameraControl} to
save or load frames to and from hard disk. You can easily access more than one
frame grabber. The program should be self-explanatory.

\texttt{cameraControl} requires Tcl/Tk libraries. If you don't have Tcl/Tk
libraries installed, you will get errors while compiling
\texttt{cameraControl}.  However, this has no effect on the camera server.
 
\subsection{MPEG support}
 
MPEG is a standard (not to say \emph{the} standard) for recording video data. Read
more about MPEG at \texttt{http://www/mpeg.org/}.
 
Recording videos involves three steps: 

\begin{enumerate}
  
\item Save the frames either with \texttt{cameraControl} or
  \texttt{cameraSaveFile( )} to a file. Let's assume this file is called
  \texttt{frames.raw}.
  
\item Extract each frame from \texttt{frames.raw} into a separate file, say
  \texttt{frame-0.raw} to \texttt{frame-N.raw}. The little script
  \texttt{\~{}bee/src/cameraServer/raw2ppm} takes care of extracting each
  frame from \texttt{frames.raw}. You have to edit \texttt{raw2ppm} to suit
  your task.
  
\item Encode frames \texttt{frame-0.raw} to \texttt{frame-N.raw} with an MPEG
  encoder to an mpeg video file. Get the MPEG encoder/decoder
  \texttt{mpeg2v12.zip} from 
  \begin{center}
    \texttt{ftp://ftp.mpeg.org/pub/mpeg/mssg/}
  \end{center}
  and install it. A simple \texttt{make} should do.  You will also need a
  parameter file. Copy one of the supplied parameter files, e. g.
  \texttt{MPEG-1.par} and modify it to your needs. Usually, you change

  \begin{itemize}
  \item the comment in the MPEG file (first line in the .par file)
  \item the name of the source file
  \item the number of frames
  \item the image size (width and height)
  \end{itemize}
  
  Take a look at \texttt{MPEG-example.par} and/or read the manual in mpeg's
  \texttt{doc/} directory.

\end{enumerate}
 
Then use 
\begin{center}
  \texttt{mpeg2encode my-parameter-file.par video.mpeg}  
\end{center}

to encode the frames. There are certain requirements: 

\begin{enumerate}
  
\item A lot of disk space. Remember that you have one .raw file and the same
  equivalent in \texttt{frame-i.raw} files. Plus: you need some temporary
  space for the MPEG encoding. Example: raw file is about 200 MB, you need an
  additional 200 MB for each extracted frame and roughly 50 MB for encoding.

\item The MPEG encoder (see above).

\item \texttt{xanim} or Netscape to display the .mpeg videos. 

\end{enumerate}

\end{document}